\documentclass[nojss]{jss}
\usepackage{url}
\usepackage[sc]{mathpazo}
\usepackage{geometry}
\geometry{verbose,tmargin=2.5cm,bmargin=2.5cm,lmargin=2.5cm,rmargin=2.5cm}
\setcounter{secnumdepth}{2}
\setcounter{tocdepth}{2}
\usepackage{breakurl}
\usepackage{hyperref}
\usepackage[ruled, vlined]{algorithm2e}
\usepackage{mathtools}
%% \usepackage{draftwatermark}
\usepackage{float}
\usepackage{placeins}
\usepackage{mathrsfs}
\usepackage{multirow}
%% \usepackage{mathbbm}
\DeclareMathOperator{\sgn}{sgn}
\DeclareMathOperator*{\argmax}{\arg\!\max}





\title{\bf Measuring Information Loss and Harmonization}

\author{Michael. C. J. Kao\\ Food and Agriculture Organization \\ of
  the United Nations}

\Plainauthor{Michael. C. J. Kao} 

\Plaintitle{Measuring Information Loss and Harmonization}

\Shorttitle{Measuring Information Loss and Harmonization}

\Abstract{ 




}

\Keywords{meta data, flag aggregation}
\Plainkeywords{meta data, flag aggregation}

\Address{
  Michael. C. J. Kao\\
  Economics and Social Statistics Division (ESS)\\
  Economic and Social Development Department (ES)\\
  Food and Agriculture Organization of the United Nations (FAO)\\
  Viale delle Terme di Caracalla 00153 Rome, Italy\\
  E-mail: \email{michael.kao@fao.org}\\
  URL: \url{https://github.com/mkao006/sws_flag}
}


\begin{document}




\section{Introduction}

%% Need to talk about the old and the new statistical working system.

%% Need to think about how to present the two problem faced and solved
%% with a single harmonized framework.
%% 

%% There are two problems, preserving the most amount of information
%% and also use of different source of information.

%% In aggregation, the symbol with the lowest weight or in turn
%% highest information content is preserved. This allow us to retain
%% the highest amount of information under the scenario where
%% information loss is inevitable.

%% Further, when it is calculated it means that the information is at
%% least this amount and potentially more. This is a restriction
%% imposed by the current system. Publishing the weights would be a
%% more favourable improvement.

%% IF the weight of production is 0.8 and area is 0.4, then the weight
%% of yield should be 0.32. Which is lower than both, we need to think
%% how to assign 0.32 to a flag under the case if there is a flag for
%% 0.4 and 0.3.

%% I think for observation status we should still use the minimum
%% rule, but it may not be the ideal solution.


International organizations such as FAO inevitably collects
information and data from various sources and channels. In order to
provide a comprehensive status of the world in their respective domain
of operation, one must utilize as much information as possible.


%% Check the dates when flags were collected, also elaborate on the
%% utilization of the flags.
Since the introduction of the statistical working system in the
1990's, symbols representing different source of information has been
recorded. This practice enabled the clerk to identify whether a value
complies with the organization policy of official values are the most
trustworthy, and also based on experience which source is more
reliable. It also underlies some of the earlier algorithm implemented
in the system whether a value should be calculated or remain as it is.


Yet, no attempt has been made to harmonize and treaet the unequal
information content of various source. Data collected from different
source and channel does not have the same information
quality. Further, the various nature of the collection may give arise
to different statistical properties. The user should not assume that
the data are of equal quality as they may and will undermine the
validity of the analysis.


The proposed methodology makes the first pursuit to provide a
systematic approach for this problem. The aim of the approach is to
allow information from separate source to represent a harmonized
picture when combined to provide the status of the world.

This paper comprises of just three sections. First, the motivation and
the problem at hand is presented and why such a framework is
necessary. Secondly, we present the theory where the fundamentals of
the principle of minimum discrimination information is
introduced. Then we illustrate the use of this framework and
demonstrate how estimators are superior when information quality are
addressed. Finally, a short conclusion and further work concludes the
paper.

\section{Motivation and Problem Statement}

We first encountered the problem in the implementation of the new
system, where historically, values which were calculated were recorded
with a symbol "C". This causes confusion as it does not represent the
source of the data but rather the process of the data. Further, the
original information it was suppose to preserve is lost. Information
loss is inevitable when several values are involved in the calculation
to provide a single figure, yet the "C" representation not only reduce
the information set to only it is calculated, it does not distinguish
the information quality whether the calculation is based on good solid
official figure or poor estimates.

The decision was taken to separate this mixing of information, and
record two flags in the new sytsem. The observation status flag will
represent the sources of the data collection, while a second flag
represent the methodologies and processes used to generate the value.

\section{Theory}


\subsection{Principle of Minimum Discriminant Information and Cross-Entropy}

Given derived information set, a new distribution $q$ should be chosen
which is as hard to discriminate from the original distribution $p$ as
possible; so that the new data produces as small an information gain
as possible.

In another word, the principle states that if we have to choose
another representation, the information set which result in the least
amount of information gain or uncertainty should be chosen.


\begin{align*}
  \mathrm{H}(P, Q) &=  \mathrm{H}(P) + D_{\mathrm{KL}}(P \| Q)\\
  \intertext{Where}
  H(P) &= -\sum_{i} {p(x_i) \log p(x_i)},\\
  D_{\mathrm{KL}}(P\|Q) &= \sum_i \log\left(\frac{p(i)}{q(i)}\right) p(i).\\
\end{align*}


The weights of different information source can then be calculated as
follow:

\begin{equation*}
  \omega_i = \left\{
  \begin{array}{l r}
    1/(1 + D_{\mathrm{KL}}(P\|Q_i)) \quad \text{if $D_{\mathrm{KL}}(P\|Q_i) \ne 0$}\\
    1 - 1e^{-5} \, \, \, \, \quad \quad \quad \quad \quad \text{if $D_{\mathrm{KL}}(P\|Q_i) = 0$}
  \end{array} \right.
\end{equation*}


Essentially, the greater the loss of information the less the
weight. This is in corcordance with our expectation that the lower the
information quality, the evidence it provide is less and thus taking
less weight in the model or analysis.


%% Read more on the theory.

\subsection{Maximum Information Perservation Flag Aggregation}

The same framework can also translate for flag aggregation, when
combining different set of information sets we need to maintain as
much information as possible. In this case, it is equivalent of taking
the flag which is classified with the lowest weight.


\section{Application}

%% Need to mention that the Empirical Kullback-Leibner is used.


\section{Conclusion and Further Work}



\end{document}
