\documentclass[nojss]{jss}
\usepackage{url}
\usepackage[sc]{mathpazo}
\usepackage{geometry}
\geometry{verbose,tmargin=2.5cm,bmargin=2.5cm,lmargin=2.5cm,rmargin=2.5cm}
\setcounter{secnumdepth}{2}
\setcounter{tocdepth}{2}
\usepackage{breakurl}
\usepackage{hyperref}
\usepackage[ruled, vlined]{algorithm2e}
\usepackage{mathtools}
\usepackage{draftwatermark}
\usepackage{float}
\usepackage{placeins}
\usepackage{mathrsfs}
\usepackage{multirow}
%% \usepackage{mathbbm}
\DeclareMathOperator{\sgn}{sgn}
\DeclareMathOperator*{\argmax}{\arg\!\max}




\title{\bf Measuring and Minimising Information Loss in Source Aggregation}

\author{Michael. C. J. Kao\\ Food and Agriculture Organization \\ of
  the United Nations}

\Plainauthor{Michael. C. J. Kao} 

\Plaintitle{Measuring and Minimising Information Loss in Source Aggregation}

\Shorttitle{Measuring and Minimising Information Loss in Source Aggregation}

\Abstract{ 




}

\Keywords{meta data, flag aggregation}
\Plainkeywords{meta data, flag aggregation}

\Address{
  Michael. C. J. Kao\\
  Economics and Social Statistics Division (ESS)\\
  Economic and Social Development Department (ES)\\
  Food and Agriculture Organization of the United Nations (FAO)\\
  Viale delle Terme di Caracalla 00153 Rome, Italy\\
  E-mail: \email{michael.kao@fao.org}\\
  URL: \url{https://github.com/mkao006/sws_flag}
}


\begin{document}




\section{Introduction}

%% Need to talk about the old and the new statistical working system.

%% Need to think about how to present the two problem faced and solved
%% with a single harmonized framework.

%% There are two problems, preserving the most amount of information
%% and also use of different source of information.

%% In aggregation, the symbol with the lowest weight or in turn
%% highest information content is preserved. This allow us to retain
%% the highest amount of information under the scenario where
%% information loss is inevitable.

%% Further, when it is calculated it means that the information is at
%% least this amount and potentially more. This is a restriction
%% imposed by the current system. Publishing the weights would be a
%% more favourable improvement.

%% IF the weight of production is 0.8 and area is 0.4, then the weight
%% of yield should be 0.32. Which is lower than both, we need to think
%% how to assign 0.32 to a flag under the case if there is a flag for
%% 0.4 and 0.3.

%% I think for observation status we should still use the minimum
%% rule, but it may not be the ideal solution.

%% The framework also enable us to decide which data source to
%% distribute. Official data are disseminated when available, but
%% under the absence of official data, this framework enable algorithm
%% to determine the best alternative representation to disseminate.


International organizations such as FAO inevitably collects
information and data from various sources and channels. In order to
provide a comprehensive status of the world in their respective domain
of operation, one must utilize as much information as possible.


%% Check the dates when flags were collected, also elaborate on the
%% utilization of the flags.
Since the introduction of the statistical working system in the
1990's, flags representing different source of information has been
recorded. This practice enabled the clerk to identify whether a value
complies with the organization policy of official values are the most
trustworthy, and also based on experience which source is more
reliable. It also underlies some of the earlier algorithm implemented
in the system whether a value should be calculated or remain as it is.


Yet, no attempt has been made to harmonize and treaet the unequal
information content of various source. Data collected from different
source and channel does not have the same information
quality. Further, the various nature of the collection may give arise
to different statistical properties. The user should not assume that
the data are of equal quality as they may and will undermine the
validity of the analysis.


The proposed methodology makes the first pursuit to provide a
systematic approach for this problem. The aim of the approach is to
allow information from separate source to represent a harmonized
picture when combined to provide the status of the world.

This paper comprises of three sections. First, the motivation and the
problem at hand is presented and why such a framework is
necessary. Secondly, we present the theory where the fundamentals of
the principle of minimum discrimination information is
introduced. Then we illustrate the use of this framework and
demonstrate how estimators are superior when information quality are
addressed. Finally, a short conclusion and further work concludes the
paper.

\section{Motivation and Problem Statement}

One of the problem first faced was to identify over the spectrum of
the data collection method, which is more reliable and should be
disseminated. The identification will enable us to decide which
statistic to disseminate allowing our user to access the most reliable
set of information.

Often an arbitrary approach is taken, some times based on experience
or perception. For example, manual estimation is current perceived as
better than algorithmic imputation. However, this may notnecessarily
be true in particularly seeing the advacement of imputation methods
which has proven to out perform human estimation. A formal assessment
framework can assist us in identifying the better source for
dissemination.


Further, when data are collected and disseminated, information were
often treated as equal and assume to behave identically. Despite being
the commonly acceptable practice, it is far from desirable. Data
collected from various source and channel can behave vastly different
depending on the tools employed. Land survery in contrast to
estimation based on satelite images can produce very different figure
for the same object of measurement.

Likewise everything we attempt to meausre, there is an uncertainty in
the measurement under the influence of noises. Whether it be passage
of time or the circumference of the world in the past, there are some
measurement uncertainty.

Various level of technological development lead to different data
collection practice. It would be infeasible to devise a standard for
which all data collection are equivalent, thus a framework to enable
users to account for data of various information quality is
indispensible.

Another common problem encountered when derived statistics were
computed, a common task in publishing official statistics.  Vast
amount of data are collected by agencies and international
organization. To effectively present the data for policy formulation
and monitoring to high level executives, aggregation and summarization
is essential. 

In the current working system, an aggregation is associated with the
flag of "C" representing that the value was computed. However, this
results in a large amount of information loss. Information on the
sources of data collection is unpreserved and lost. 

%% This in term can be translated to information quality of a measurement
%% of interest. The lower the measurement uncertainty, the higher the
%% information quality.

To resolve this problem, an entropy approach was taken to quantify
information in order to measure the information loss and further
minimize the information loss in the process of aggregation of
information.

\section{Theory}

Information theory has its root in communication, where Claude
E. Shannon laid the foundation of the field with the paper "The
Mathematical Theory of Communication". 
%% Elaborate on the history of information theory.

The quantification of information enables us to determine which source
to disseminate as we are restricted to disseminate only one measurable
value.

In order to quantify the information, we assume data collected by
official sources are correct and perceive it as the signal under
perfect condition. This is in concordance with the organization's
policy where official sources are taken as golden measure. Based on
this assumption, we can proceed with calculating the information loss
when data from a different source is used in the absence of official
data.


\subsection{Principle of Minimum Discriminant Information and Cross-Entropy}

Given derived information set, a new distribution $q$ should be chosen
which is as hard to discriminate from the original distribution $p$ as
possible; so that the new data produces as small an information gain
as possible.

In another word, the principle states that if we have to choose
another representation when official figures are unavailable, the
information set which result in the least amount of information gain
or uncertainty should be chosen.

Provided this, we can choose between which information is under less
influence of noise and better represents the true quanity of
measurement.


\begin{align*}
  \mathrm{H}(P, Q) &=  \mathrm{H}(P) + D_{\mathrm{KL}}(P \| Q)\\
  \intertext{Where}
  H(P) &= -\sum_{i} {p(x_i) \log p(x_i)},\\
  \intertext{and,}
  D_{\mathrm{KL}}(P\|Q) &= \sum_i \log\left(\frac{p(i)}{q(i)}\right) p(i).\\
\end{align*}

The quantity $\mathrm{H}(P, Q)$ is known as cross-entropy and it is
the difference between the entropy of the true distribuion $H(p)$ and
the information lost measured by the Kullbak-Leibner
$D_{\mathrm{KL}}(P\|Q)$. 

%% In our case, the entropy of $p$ is equivalent in all comparison cases
%% when it is applied to the official figures. Thus, we can simply
%% compute the the information loss of the Kullbak-Leibner
%% $D_{\mathrm{KL}}(P\|Q)$.

In practical cases, the empirical Kullback-Leibner is
calculated. Values are binned and the relative frequency is used to
approximate the real density $p(x)$ and $q(x)$. That is, the index $i$
is not a single point, rather it is a binned density.

This quantity inform us about the information loss associating with
the employ of an alternative data source. Thus, we would choose the
representation in which the information loss is minimised.

\subsection{Maximum Information Perservation Flag Aggregation}

After measuring the information, we can apply to the process of
aggregation when the maximum amount of information is preserved under
aggreagtion.

In a typical aggregation scenario or computation of derived
statistics, we have:

\begin{equation}
S_{agg} = f(S_{1}, S_{2}, \cdots, S_{n})
\end{equation}

Since we are restricted to a finite list of available symbol
representing different information sources and only one symbol can be
used; the loss of information is unavoidable under aggregation. The
goal here is to choose a function $f$ which conveys maximum
information.

Using entropy, we can quantify the information associated with each
data source. The equation then becomes:

\begin{equation}
S_{agg} = F(\max(H(S_{1}), H(S_{2}), \cdots, H(S_{n})))
\end{equation}

Where $F$ represents the conversion of the entropy to the character
symbol representing the data source. This allows us to preserve as
much information as possible under aggregation.


\section{Application}

The value of cross-entropy itself is not very useful for users, thus
we have created what we call the information weights for users to
utilize for modelling and subsequent analysis.

\subsection{Information Weight}
The weights of different information source can be calculated based on
the Kullback-Leibner as follow:

\begin{equation*}
  \omega_i = \left\{
  \begin{array}{l r}
    1/(1 + D_{\mathrm{KL}}(P\|Q_i)) \quad \text{if $D_{\mathrm{KL}}(P\|Q_i) \ne 0$}\\
    1 - 1e^{-5} \, \, \, \, \quad \quad \quad \quad \quad \text{if $D_{\mathrm{KL}}(P\|Q_i) = 0$}
  \end{array} \right.
\end{equation*}


Essentially, the greater the loss of information the less the
weight. This is in corcordance with our expectation that the lower the
information quality, the evidence it provide is less and thus taking
less weight in the model or analysis.


%% Read more on the theory.






%% Need to mention that the Empirical Kullback-Leibner is used.


\section{Conclusion and Further Work}



\end{document}
